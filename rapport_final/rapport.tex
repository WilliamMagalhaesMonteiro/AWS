\documentclass[11pt,a4paper]{article}
\usepackage{hyperref}

\date{}
\title{Rapport projet AWS - Pictionary}
\author{ Hugo Chanas \\ William Magalhaes Monteiro \\ Asmaa Ouguenoune \\ Christian Knayzeh }

\begin{document}
    \maketitle
    \section{Présentation du projet}
        \subsection{L'idée de base}
            \paragraph{}
            L'idée de base du projet est de proposer un jeu de type Pictionary.
            Un joueur doit faire deviner un mot en le dessinant et les autres doivent le trouver le plus vite possible.
            Le projet est très inspiré de \href{https://skribbl.io/}{Skribbl.io}.
            \paragraph{}
            Il nous faut donc au moins la possibilité de dessiner, avec le partage du dessin en temps réel,
            et un chat où les utilisateurs peuvent tenter de deviner le mot.
        \subsection{Les technologies que l'on a utilisées}
            \paragraph{}
            Pour la partie dessin, nous aurions pu utiliser l'API de base pour dessiner sur des canvas HTML,
            mais nous avons préféré utiliser le framework Konva.js car il plus facile à prendre en main et à utiliser,
            et apporte plus de fonctionnalités.
            \paragraph{}
            Pour gérer la communication, nous avons choisi Socket.io, qui permet de gérer la communication en temps réel,
            aussi bien pour le dessin que pour le chat. Pour le reste, nous avons utilisé Express.
            \paragraph{}
            Nous utilisons aussi Express session pour gérer l'authentification et les sessions.
            Les comptes sont gérés avec SQLite.
            Nous avons commencé avec une base de données MySQL mais avons décidé de changer pour simplifier le projet et (surtout) l'hébergement (finalement sur Glitch).
    \section{Ce que l'on a fait - Description du projet}
        \subsection{Authentification et base de données}
            \paragraph{}
            Notre application ne permet qu'aux joueurs ayant un compte d'accéder au jeu.
            Il existe donc trois pages autres que celle du jeu :
            \begin{itemize}
                \item Une page d'accueil,
                \item Une page d'authentification pour se connecter,
                \item Une page de création de compte.
            \end{itemize}
            \paragraph{}
            Pour gérer les réponses du serveur sur ces deux dernières,
            on utilise le moteur pug.
            On peut ainsi afficher des messages d'erreurs sans avoir besoin de script.
            On affiche donc un message rouge en bas de la page lorsqu'un nom d'utilisateur ou mot de passe est invalide,
            ou encore si le nom choisi est déjà utilisé, etc.
            \paragraph{}
            Dès qu'une requête de connexion ou de création est reçue,
            le serveur interroge l'instance SQLite et ajoute le compte ou autorise la connexion en conséquence.

            La base de données ne contient qu'une table pour les comptes, avec un nom d'utilisateur, un mot de passe haché,
            et une valeur aléatoire utilisée comme padding sur le mot de passe avant le hachage.
            \paragraph{}
            Pour la liste des mots à dessiner qui peuvent être proposés aux joueurs,
            elle est enregistrée dans un simple fichier.
            En effet, comme la liste est fixée, et non modifiable par un utilisateur,
            il n'est pas nécessaire de l'ajouter à la base.
        \subsection{Partie jeu}
            \subsubsection{Dessin}
                \paragraph{}
                Pour la partie dessin, on utilise des objets Konva.
                Pour chaque outil disponible, une série de 'binds' est définie,
                et appliquée lorsque l'utilisateur clique sur le bouton qui correspond.
                Chaque nouvel élément du dessin est un nouvel objet Konva qui est ajouté au Layer.
                \paragraph{}
                Pour le crayon ou la gomme, à chaque nouveau clic un objet Konva.Line est créé.
                Tant que le clic est enfoncé, et à chaque mouvement de la souris, un point est ajouté à la ligne.
                Un tel objet représente en effet une suite de segments reliés entre eux.
                Les attributs de l'objet sont définis en fonction des choix de l'utilisateur pour l'épaisseur et la couleur.
                \paragraph{}
                Pour l'outil `rond', un disque est ajouté à chaque nouveau clic.
                Cet outil très rudimentaire a été ajouté en prévision d'une modification ou d'une amélioration,
                mais qui n'a finalement pas été faite.
                \paragraph{}
                L'implémentation de l'outil remplissage était de loin la plus difficile.
                En effet, il a fallu accéder au contexte 2D du canvas pour obtenir la matrice de pixel,
                et appliquer un algorithme de remplissage par diffusion.
                Après avoir obtenu une nouvelle matrice de pixels correspondant à la zone à colorier,
                une image bitmap est crée, et est utilisée pour initialiser un objet Konva.Image.

                Cet outil est assez important dans une application de dessin,
                et aurait manqué si il n'avait pas été ajouté.
                \paragraph{}
                Pour les boutons `undo' et `redo' (les flèches), un simple nombre entier est incrémenté ou décrémenté.
                Ce nombre représente le nombre d'objet de la liste qui sont affichés à l'écran.
                Lorque le compteur diminue, un objet est effacé de la zone de dessin.
                A l'inverse le dernier objet a avoir été enlevé est réajouté quand le compteur augmente.
                Si l'utilisateur dessin un nouvel élément alors que certains on été effacés,
                ceux-ci sont définitivement supprimés,
                et le comtpeur prend comme valeur la taille du tableau d'objets affichés.

                Le bouton `suppression' est quand à lui plus simple, il supprime simplement tous les objets du Layer.
            \subsubsection{Communications}
                \paragraph{}
                Une fois sur la page du jeu, toutes les communications entre les clients et le serveur sont gérées avec Socket.io.
                \paragraph{}
                Lorsque le serveur désigne un ou plusieurs dessinateurs, ceux-ci sont placés dans une `room',
                ce qui permet de leur envoyer des messages ciblés.
                Par exemple, le mot qu'ils doivent faire deviner n'est envoyé qu'à eux.
                Les autres reçoivent en effet une chaîne de caractères avec les lettre remplacées par des tirets bas (comme au pendu).

                Une autre room existe où sont placés les utilisateurs qui doivent deviner le mot et qui ne l'ont pas encore trouvé.
                Les messages du chat envoyés par les dessinateurs et ceux qui ont devinés ne sont ainsi pas visibles par ceux qui cherchent.
                \paragraph{}
                Dès qu'un dessinateur modifie le dessin, l'information est transmise au serveur, qui les transmet à son tour aux autres utilisateurs.
                Pour les traits et les disques,
                des coordonnées et une couleur sont transmises ce qui permet aux autres utilisateurs de créer des objet parfaitement indentiques de leur côté.
                \paragraph{}
                Pour le remplissage, c'est un peu plus compliqué.
                Transmettre la matrice de pixels entière serait en effet assez lourd ($\approx$ 1Mo par image).
                Il est assez simple de réduire le volume de données, et on a utilisé deux solutions.
                
                Dans un premier temps, le remplissage n'a qu'une seule couleur.
                On peut donc transmettre la couleur en parallèle du tableau,
                et réduire l'information de chaque pixel à un booléen.
                Pour des raisons techniques, les pixels sont représentés par des octets,
                bien qu'ils ne puissent prendre que deux valeurs.

                Cela donne un tableau de 0 et de 1.
                Pour réduire encore le volume de données,
                chaque suite de 1 est remplacée par deux nombres :
                la position du premier 1 dans le tableau, et la longueur de la suite horizontale.
                Il ne reste plus qu'à transmettre ces informations pour reconstituer l'image complète.
                \paragraph{}
                L'ensemble des modifications apportées au dessin sont traités et enregistrées par le serveur,
                y compris undo, redo et la suppression.
                Ansi, lorsqu'un nouvel utilisateur rejoint le jeu en cours de route,
                la pile d'objets dessinés lui est envoyée pour qu'il puisse afficher le dessin tel qu'il est à cet instant.
                \paragraph{}
                Le déroulement du jeu est planifié par le serveur.
                Au début de chaque manche, le serveur désigne donc le ou les dessinateurs,
                et envoie un message à ceux-ci contenant le mot qu'ils doivent faire deviner.
                Il attend ensuite de recevoir la confirmation de tous les dessinateurs (quand il cliquent sur `commencer') pour signaler
                à tout le monde que la manche commence.
                \paragraph{}
                A chaque fois qu'un joueur faisant partie de ceux qui devinent envoie un message dans le chat,
                il est vérifié si il correspond au mot à deviner ou non pour savoir si il a gagné.
                \paragraph{}
                Lorque le temps est écoulé ou que tous les joueurs ont trouvé le mot, la manche se termine,
                les points sont calculés en fonction de l'ordre dans lequels les joueurs ont deviné,
                et une nouvelle manche commence.
                \paragraph{}
                Pour choisir les dessinateurs, un premier est désigné de façon cyclique (l'un après l'autre),
                puis, s'il y en a d'autres, de façons aléatoire.
                Ainsi, un tour de jeu correspond à un cycle,
                ce qui permet à chaque joueur de dessiner au moins une fois par tour.
                Lorsque le nombre de tours joué atteint le nombre spécifié au début par le créateur de la partie,
                le jeu se termine et le joueur à avoir marqué le plus de points est vainqueur.

    \section{Ce que l'on a appris}
        \subsection*{Travailler en groupe (avec les 3 rôles à permuter par semaine)}
            \paragraph{}
            Cela a sûrement été la partie la plus difficile.
            En effet, nous ne nous sommes pas rencontrés souvent,
            et avons donc dû communiquer presque exclusivement sur un serveur discord...
            \paragraph{}
            Le fait d'interchanger les rôles chaque semaine était une experience assez intéressante,
            nous forçant tous à fournir une partie du travail,
            et à apporter notre vision au projet.
        \subsection*{Gestion des requêtes des clients, réponses, redirections}
            \paragraph{}
            Nous avons appris comment fonctionne une application web classique,
            avec un ensemble de pages dans lesquelles naviguent des utilisateurs,
            envoyant régulièrement diverses requêtes au serveur.
            \paragraph{}
            Nous avons utilisé des middlewares pour gérer l'authentification,
            les sessions et l'interrogation de la base de données.
        
        \subsection*{Gestion d'une petite base de données}
            \paragraph{}
            Pour enregistrer les comptes,
            nous avons dû apprendre à manipuler une petite base de données dans le script du serveur.
        \subsection*{Écriture de script javascript complexe pour une application}
            \paragraph{}
            Les scripts côté serveur et client qui gère la partie jeu sont assez longs et complexes.
            Nous avons donc lors de leur écriture beacoup progressé en javascript.
            \paragraph{}
            Certains aspects notamment comme l'outil remplissage avec des matrices de pixels et des images bitmap 
            ont été particulièrement instructifs.
        \subsection*{Communication en temps réel entre un serveur et plusieurs clients}
            \paragraph{}
            Pour obtenir un jeu fonctionnel en temps réel, nous avons dû apprendre à bien manipuler Socket.io.
            Cette partie est en effet une nouveauté,
            bien différente de la programmation séquentielle à laquelle nous sommes habitués.
        \subsection*{Sécuriser une base de données contenant des mots de passe.}
            \paragraph{}
            Nous avons pu appliquer directement les enseignements de notre maître en sécurité avec notre base de données.
            Cela nous a permis de nous initier à l'utilisation de la cryptographie dans les applications web en javascript.
    \section{Ce que l'on pourrait ajouter ou améliorer}
        \subsection*{Fonctionnalités diverses du jeu}
            \begin{itemize}
                \item Ajouter et améliorer des outils
                
                L'outil `rond' est particulièrement discutable,
                et faisait partie du projet d'ajouter des formes prédéfinies diverses,
                pouvant être placées avec précisions, bougées, etc.

                Konva permet de manipuler des objets graphiques facilement,
                et nous aurions pu en profiter pour facilement implémenter ces fonctionnalités.
                \item Ajouter un sélecteur de couleurs
                
                Notre jeu ne dispose que d'une quantité limitée de couleurs parmi lesquelles choisir.
                Il serait pratique pour les dessinateurs de pouvoir choisir une couleur dans un sélecteur dédié.

                Cependant, comme il s'agit aussi d'un jeu de vitesse,
                la majorité des joueurs se contentent des couleurs prédéfinies.
                Pour cette raison, nous nous sommes contentés de quelques couleurs.
                \item Le déroulement du jeu
                
                Un assez grand nombre de petites améliorations peuvent encore être faites sur le jeu.

                Par exemple, la façon de choisir les dessinateurs quand il y en a plusieurs repose en partie sur l'aléatoire.
                Le mieux serait de choisir de façon déterministe, en s'assurant que tout le monde dessine autant de fois,
                et en essayant de créer des groupes différents à chaque fois.

                Le méthode pour compter les points n'est pas très avancées non plus, et pourrait largement être améliorée.
            \end{itemize}
        \subsection*{Page dédiée à la modification de la liste de mots par des admins.}
            \paragraph{}
            Nous avons une liste de mots statique.
            Elle n'est donc pas incluse dans la base de données,
            et ne peut être modifiée que par l'administrateur du serveur.
            \paragraph{}
            Une bonne amélioration à apporter serait de l'y ajouter,
            et de créer une page accessible par des utilisateurs ayant un rôle admin qui leut permettent d'ajouter ou d'enlever des mots.
            Nous aurions alors besoin d'ajouter des rôles et privilèges aux utilisateurs,
            ce qui peut être un aspect intéressant à apprendre.
        \subsection*{Autoriser aux utilisateurs à ne pas créer un compte}
            \paragraph{}
            Les utilisateurs sont forcés de créer un compte pour jouer au jeu.
            Cependant, rien n'est associé au compte à part le nom d'utilisateur et le mot de passe.
            Nous pourrions donc simplement demander à des utilisateurs sans compte d'entrer un nom
            qui serait utilisé seulement pendant sa session de jeu.
            \paragraph{}
            Il serait aussi possible par la même occasion d'ajouter des données liées aux comptes,
            comme des statistiques, visibles sur une page dédiée.
            Cela justifierait vraiment la création des comptes pour les utilisateurs.


\end{document}
