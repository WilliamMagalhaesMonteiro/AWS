\documentclass[11pt,a4paper]{article}


\date{}
\title{Rapport projet AWS - Pictionary}
\author{ Hugo Chanas \\ William Magalhaes Monteiro \\ Asmaa Ouguenoune \\ Christian Knayzeh }

\begin{document}
    \maketitle
    \section{Présentation du projet}
        \subsection{L'idée de base}
            \paragraph{}
            L'idée de base du projet est de proposer un jeu de type pictionary.
            Un joueur doit faire deviner un mot en le dessinant et les autres doivent trouver le plus vite possible.
            Le projet est très inspiré de Skribbl.io.
            \paragraph{}
            Il nous faut donc au moins la possibilité de dessiner, avec le partage du dessin en temps réel,
            et un chat où les utilisateurs peuvent tenter de deviner le mot.
        \subsection{Les technologies que l'on a utilisées}
            \paragraph{}
            Pour la partie dessin, nous aurions pu utiliser l'API de base pour dessiner sur des canvas HTML,
            mais nous avons choisi d'utiliser le framework Konva pour nous simplifier la tâche, et obtenir plus de fonctionnalités.
            \paragraph{}
            Pour gérer la communication, nous sommes partis sur socket.io, qui permet de gérer la communication en temps réel,
            aussi bien pour le dessin que pour le chat.
            \paragraph{}
            Nous utilisons aussi Express et Express session pour gérer l'authentification et les sessions.
            Les comptes sont gérés avec SQLite.
            Nous avons commencé avec une base de données MySQL mais avons décidé de changer pour simplifier le projet et (surtout) l'hébergement (finalement sur Glitch).
    \section{Ce que l'on a fait - Description du projet}
        \subsection{Authentification et base de données}
        \subsection{Partie jeu}
    \section{Ce que l'on a appris}
        \subsection{Gérer des comptes et l'authentification}
        \subsection{Communication en temps réel}
\end{document}
